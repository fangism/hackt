% "chapter/instance.tex"
%	$Id: instance.tex,v 1.3 2005/05/06 02:51:22 fang Exp $

\chapter{Instances}
\label{sec:instance}

%%%%%%%%%%%%%%%%%%%%%%%%%%%%%%%%%%%%%%%%%%%%%%%%%%%%%%%%%%%%%%%%%%%%%%%%%%%%%%%
\subsubsection{Virtual vs. physical instances}

Hierarchical names handled by the front-end correspond to virtual instances.  
Virtual instances are an indirection to physical instances.  
Multiple virtual instances may alias to the same physical instance.  
(Nomenclature: virtual instances are currently called instance aliases
	in the compiler implementation --- get the names straight!)

Only after all connections have been unrolled and established are the 
unique physical instances created.  
This step of unique instance creation is the final step of the 
front-end compiler.  

\tbf{Implementation:}
Concern: this expanded form is only useful to a subset of the tools, 
such as simulators and analyzers.  
The expanded form is going to be large, which is undesirable for 
the tools that don't need it.  
Consider: The final unique creation phase should be tool-specific, 
but provide a library function for it.  

%%%%%%%%%%%%%%%%%%%%%%%%%%%%%%%%%%%%%%%%%%%%%%%%%%%%%%%%%%%%%%%%%%%%%%%%%%%%%%%
\section{Instance Collections}
\label{sec:instance:collection}

Every instance name may refer to either a single (scalar, 0-dimensional)
virtual instance, or a higher-dimension array (up to 4) of virtual instances.  
When we refer to instance collections, we also include the case for
0-dimensional instances.  

The contents of an instance collection is summarized as follows:

\begin{enumerate}
\item pointer: back-reference to parent namespace
\item string: name of this instance collection
\item map of virtual instance that are 
	\emph{actually unrolled} during the unique creation phase.  
	Each individual virtual instance contains
	the index with which it is associated.
	The virtual instances sequence is stored as a list of pointers.  
\end{enumerate}

Prior to the creation phase, the map of index-instance pairs may be empty.  

In addition, the following subtypes have additional fields stored with 
the instance collections:

\begin{itemize}
\item Process: contains an additional process type reference.  
\item Channel: contains an additional channel type reference.  
\item Data (excluding int and bool): 
	contains an additional data type reference.  
\item int: contains a integer (4B) width parameter
\item bool: nothing else
\end{itemize}

For instance collections that reside outside of definitions, 
the types referenced by the instance collections must be resolved
to canonical definitions i.e, they may not reference typedefs.  
Instance collections residing inside definitions may reference typedefs.  

NOTE: the type reference contained by the collection must have
strict template arguments specified, but need not have
relaxed template arguments specified.  
If the relaxed arguments are specified, then they apply
to all instance members of the collection, 
and the members are forbidden from supplying relaxed template arguments.  
If the relaxed arguments are not specified, then the 
instance members (see Section~\ref{sec:instance:virtual})
\emph{must} supply relaxed template arguments to complete their types.  

Scalars instance collections simply replace the sequence of 
virtual instance with a single virtual instance.  
The scalar collection's type reference must specify relaxed template
arguments, if applicable.  

%%%%%%%%%%%%%%%%%%%%%%%%%%%%%%%%%%%%%%%%%%%%%%%%%%%%%%%%%%%%%%%%%%%%%%%%%%%%%%%
\section{Virtual Instances}
\label{sec:instance:virtual}

All virtual instances have the following data in common:

\begin{enumerate}
\item multidimensional index used to associate with the collection
	of which it is a member (only if non-scalar)
\item pointer: alias connection to the next of its kind.
	(Connections are maintained as circular linked lists
	of virtual instances of the same type.  
	This way, all aliases are reachable from each other.)
\item pointer: to the unique physical instance 
	(spawned only after the unique creation phase)
\item pointer: back-reference to the collection of which this is a member
\item (proposed): relaxed template parameters, 
	which may be different for each member of the collection
	(with the same strict parameters, of course)
\end{enumerate}

The only distinction for the various subtypes of virtual instances is that
the pointers are statically typed to their own respective type.
(Processes point to process collections, physical processes, etc.)

\begin{comment}
\subsection{Process VI}
\label{sec:instance:virtual:proc}

\subsection{Channel VI}
\label{sec:instance:virtual:chan}

\subsection{Data VI}
\label{sec:instance:virtual:data}
\end{comment}

%%%%%%%%%%%%%%%%%%%%%%%%%%%%%%%%%%%%%%%%%%%%%%%%%%%%%%%%%%%%%%%%%%%%%%%%%%%%%%%
\section{Physical Instances}
\label{sec:instance:physical}

These are only created by the expansion phase of the compiler, 
which follows the final type-check phase.  
(There is currently a debate whether the expansion phase should
be done by default.)  

%%%%%%%%%%%%%%%%%%%%%%%%%%%%%%%%%%%%%%%%%%%%%%%%%%%%%%%%%%%%%%%%%%%%%%%%%%%%%%%
\section{Instance References}
\label{sec:instance:reference}

Instance references contain a pointer to an instance collection
and an optional index expression.  

% % % % % % % % % % % % % % % % % % % % % % % % % % % % % % % % % % % % % % % %
\subsection{Member Instance References}
\label{sec:instance:reference:member}

Blah, blah, blah...


