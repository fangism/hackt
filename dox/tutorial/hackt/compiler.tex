% "hackt/compiler.tex"
%	$Id: compiler.tex,v 1.1 2006/05/26 07:27:01 fang Exp $

\chapter{Compiler}
\label{sec:compiler}

TODO: figure of compile flow and phases.

\section{Compile}
\label{sec:compiler:compile}

The first compile phase produces a parsed and partially checked object
file given an input text (source) file.  

Usage:
\binhackt \ttt{compile} \ttt{[}\tit{options}\ttt{]} \tit{source} 
	\ttt{[}\tit{object}\ttt{]}

The source file is a text file in the \HAC\ language.  
The object file, if given, is the result of the compile.  
If the object file is omitted, the program just reports the result
of complation without producing an object file.  

Options:
\begin{itemize}
\item \ttt{-h}: show usage
\item \ttt{-I} \tit{path} : adds include path for importing other source files (repeatable)
\item \ttt{-d}: produces text dump of compiled module, like \ttt{objdump} in Section~\ref{sec:diagnostics:objdump}
\item \ttt{-f} \tit{opt} : general compile flags (repeatable)
\begin{itemize}
	\item \ttt{dump-include-paths}: 
		dumps \ttt{-I} include paths as they are processed
	\item \ttt{dump-object-header}: 
		(diagnostic) dumps persistent object header before saving
	\item \ttt{no-dump-include-paths}: 
		suppress feedback of \ttt{-I} include paths
	\item \ttt{no-dump-object-header}: 
		suppress persistent object header dump
\end{itemize}
\end{itemize}

\ttt{haco} is provided as a single-command alias to \ttt{hackt compile}.  

\section{Unroll}
\label{sec:compiler:unroll}

\ttt{hacunroll} is provided as a single-command alias to \ttt{hackt unroll}.  

\section{Create}
\label{sec:compiler:create}

\ttt{haccreate} is provided as a single-command alias to \ttt{hackt create}.  

\section{Allocate}
\label{sec:compiler:alloc}

\ttt{hacalloc} is provided as a single-command alias to \ttt{hackt alloc}.  

Before you start worrying about having to keep track of compile phases, 
one feature is that the unroll, create, and allocate compile phases 
automatically run the necessary prerequisite phases where needed.  

\section{Conventions}
\label{sec:compiler:comventions}



\section{Examples}
\label{sec:compiler:examples}

In this section, we use the following source `\ttt{foo.hac}' 
as our input example.  

\begin{verbatim}
defproc inv(bool a, b) {
prs {
	a	-> b-
	~a	-> b+
}
}

bool x, y;
inv Z(x, y);
\end{verbatim}

