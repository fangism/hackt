% "hackt/legacy.tex"
%	$Id: legacy.tex,v 1.3 2006/09/28 03:36:16 fang Exp $

\chapter{Legacy Compatibility}
\label{sec:legacy}

NOTE: this section is somewhat of redundant with the \ttt{cast2hac} 
directory documentation.  
Please refer to \ttt{cast2hac.pdf} for a guide on migrating
to the new \hackt\ tools.  

This section is only useful to those who have used the legacy \CAST\ tools.  
We provide some tool commands for use with legacy \CAST\ tools.  
The aim is to provide a bridge from old tools to \hackt.  

\section{CAST Flatten}
\label{sec:legacy:cflat}

The old \CAST\ tool chain uses flattened text files as input to other tools.  
We provide similar functionality with \binhackt's \ttt{cflat} command, 
which is also installed under the alias \ttt{hflat}.  

Usage:
\binhackt\ \ttt{cflat} \tit{mode} \ttt{[}\tit{options}\ttt{]} \tit{in-object}

Instead of reading in the source file directly, it reads a compiled
object file.  
(Later, we may add an option to read a source file directly.)
If the object file is not already in the allocated state
(Section~\ref{sec:compiler:alloc}), then it will automatically
invoke the allocation phase before doing its real work.  
The options and modes are described in 
Section~\ref{sec:legacy:cflat:options}.  

Starting with our example from Section~\ref{sec:compiler:examples}, 
we compile \ttt{inv.haco} first.  

\medskip
\begin{verbatim}
hackt compile inv.hac inv.haco
\end{verbatim}

We then produce flattened text output with the command:

\medskip
\begin{verbatim}
hackt cflat prsim inv.haco
\end{verbatim}

\noindent
which results the following output, suitable for legacy \ttt{prsim}:

\begin{verbatim}
"x" -> "y"-
~"x" -> "y"+
= "x" "Z.a"
= "y" "Z.b"
\end{verbatim}

This can be piped directly into \ttt{prsim} or saved to a file for later use.  

\subsection{CFLAT options}
\label{sec:legacy:cflat:options}

\binhackt\ \ttt{cflat} provides convenient and fine-grain control
over the output text format.  
Options can be divided into two categories, \emph{modes} and \emph{flags}.  
Flags control individual traits of the output format, 
whereas modes are presets of traits, named after specific tools. 
The presets are set to emulate the formats expected by the legacy tools
as closely as possible.  
Currently, the following list of modes is supported:

\begin{itemize}
\item \ttt{ADspice}
\item \ttt{Aspice}
\item \ttt{LVS}
\item \ttt{alint}
\item \ttt{aspice}
\item \ttt{check}
\item \ttt{connect}
\item \ttt{default}
\item \ttt{ergen}
\item \ttt{lvs}
\item \ttt{prlint}
\item \ttt{prs2tau}
\item \ttt{prsim}
\item \ttt{wire}
\end{itemize}

TODO: make table summarizing the flags implied by each preset mode.  

The following list of flags is supported:

\begin{itemize}
\item \ttt{SEU} enable single-event-upset mode, 
	applicable only to specific tools
\item \ttt{check-mode} silences cflat output while traversing hierarchy
\item \ttt{connect-connect} alias-style: \ttt{connect x y}
\item \ttt{connect-equal} alias-style: \ttt{= x y}
\item \ttt{connect-none} suppresses all printing of aliases
\item \ttt{connect-wire} alias-style: \ttt{wire x y}
\item \ttt{no-connect} same as \ttt{connect-none}
\item \ttt{dsim-prs} wraps production rules in: \ttt{dsim \lcbrace\ ... \rcbrace}
\item \ttt{exclude-prs} excludes production rules from output
\item \ttt{include-prs} includes production rules in output
\item \ttt{quote-names} wraps all node names in ``quotes''
\item \ttt{self-aliases} includes aliases \tit{x = x}
\item \ttt{sizes} prints rule literals with 
	\labracket\tit{size}\rabracket\ specifications
\item \ttt{wire-mode} accumulate aliases in the form: \ttt{wire (x,y,...)}
\end{itemize}

All options except the \ttt{connect-\asterisk} options may also be 
prefixed with \ttt{no-} for negation, e.g. \ttt{-f no-sizes} disables
printing of sized production rule literals.  

