% "cast2hac/definitions.tex"
%	$Id: definitions.tex,v 1.3 2006/06/27 02:33:10 fang Exp $

\section{Definitions}
\label{sec:definitions}

Definitions use the keywords \defproc, \defchan, and \defdata.  
The old \CAST\ keyword \ttt{define} can simply be replaced with \defproc.  

TODO: write more on newer port restrictions...

\subsection{Ports}
\label{sec:definitions:ports}

In \CAST, port declarations were allowed to be sparse, 
whereas in \hac, only dense declarations are allowed.

The following \CAST\ definitions were legal:

\begin{verbatim}
define foo()(node x[1..3]) { }
define bar()(node x[0..2]) { }
\end{verbatim}

However, in \hac, there is no equivalent to declaring port instances
that start with non-zero indices.  
Note, that in \CAST, the first set of parentheses in each definition
are reserved for template parameterization.  
Thus, the above definition of \ttt{foo} cannot be expressed in \hac, 
whereas the definition of \ttt{bar} could be rewritten:

\begin{verbatim}
defproc bar(bool x[3]) { }
\end{verbatim}

See Section~\ref{sec:arrays} for more on dense and sparse arrays 
and their declarations.  
Section~\ref{sec:templates} describes how to generalize definitions
using template parameters.  


