% "cast2hac/templates.tex"
%	$Id: templates.tex,v 1.2 2006/06/03 05:52:29 fang Exp $

\section{Templates}
\label{sec:templates}

\CAST\ supported parameterized definitions and types, also known as templates.
\hac\ supports templates using C++-like template syntax.  
Templates are useful for defining a family types with 
highly regular differences.  
Such genearlizations often occur in definitions whose 
port sizes vary trivially.  
The most notable differences from \CAST\ templates and \hac\ templates
is the use of angle brackets around template arguments 
and parameter declarations, and the grammatic location of the 
template signature.  

In \CAST, a parameterized definition might look like this:

\begin{verbatim}
define foo(int N) (node[N] x) { }
\end{verbatim}

The equivalent definition in \hac\ would be:

\begin{verbatim}
template <pint N>
defproc foo (bool x[N]) { }
\end{verbatim}

\CAST\ and \hac alike support parameter-dependent template parameters.

\begin{verbatim}
define array(int N, D[N]) () { }
\end{verbatim}

\noindent
would be written in \hac\ as:

\begin{verbatim}
template <pint N, D[N]>
defproc array() { }
\end{verbatim}

Definitions are not the only templatable entities.  
Section~\ref{sec:typedefs:templates} discusses how typedefs 
can be templated in \hac.  

\subsection{Relaxed Templates}
\label{sec:templates:relaxed}

