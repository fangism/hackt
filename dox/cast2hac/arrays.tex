% "cast2hac/arrays.tex"
%	$Id: arrays.tex,v 1.1 2006/05/27 04:52:55 fang Exp $

\section{Arrays}
\label{sec:arrays}

In \CAST, the dimensions were declared between the type-identifier
and the instance-identifier.  
Arrays in \hac\ are syntactically C-style, where the dimensions of the 
array follow the array's identifier.  

Sparse arrays are also supported but with different syntax.  
In \hac, sparse arrays are \emph{always} declared using a range
expression in the indices.  
e.g. \ttt{inv x[1..1];} declares a sparse 1D array populated at index 1.  
(Yes, some of you may find this inconvenient.)

Table~\ref{tab:arrays} shows some examples of equivalent declarations in
\CAST\ and \hac, where \ttt{inv} is defined as a type.  

\begin{table}[ht]
\begin{center}
\caption{Examples of dense and sparse array declarations}
\label{tab:arrays}
\begin{tabular}{|c|c|l|}
\hline
\CAST & \hac & meaning \\ \hline \hline
\ttt{inv[2] x;} & \ttt{inv x[2];} & 1D array with indices 0..1 \\ \hline
\ttt{inv[2] x, y;} & \ttt{inv x[2], y[2];} & 2 1D arrays with indices 0..1 \\ \hline
\ttt{inv[2][3] x;} & \ttt{inv x[2][3];} & 2D array with indices 0..1,0..2 \\ \hline
\ttt{inv x[2];} & \ttt{inv x[2..2];} & sparse 1D array indexed 2 only \\ \hline
\ttt{inv x[2..4];} & \ttt{inv x[2..4];} & sparse 1D array indexed 2..4 \\ \hline
\ttt{inv x[2], x[4];} & \ttt{inv x[2..2], x[4..4];} & sparse 1D array indexed 2, 4 only \\ \hline
\end{tabular}
\end{center}
\end{table}

A common pitfall is to declare sparse index of an array and pass port
connections in the same statement, such as: \ttt{inv w[2](x, y, z);}.
This illegal statement tries to declare an \emph{array} indexed 0..1, 
and connect both instances with the same port parameters.  
In \hac, one cannot declare a collection and connect its ports
in the same statement as in \CAST, however, one may declare a scalar instance
and connect its ports in the same statement.  
The proper way to instantiate and connect a sparse instance
is to use a sparse range, just like in a sparse declaration:
\ttt{inv w[2..2](x, y, z);}.  

