% "cast2hac/CHP.tex"
%	$Id: CHP.tex,v 1.3 2006/06/27 02:33:09 fang Exp $

\section{CHP}
\label{sec:chp}

The CHP language is slightly different in \hac\ than in \CAST.  

\subsection{Lexical Conventions}
\label{sec:chp:lex}

In \hac, \ttt{\chpendprobsel} is a token for ``end probabalistic selection,''
thus, if the dividend of a modulus expression is indexed
(ending with a \rbracket), then an extra space is required.
For example, \ttt{x[i]\modulus{}2} must be written as
\ttt{x[i]~\modulus{}2}.

\subsection{Grammar}
\label{sec:chp:grammar}

Nondeterministic selections are delimited by \ttt{:} in \hac, 
whereas they were delimited by \ttt{|} in \CAST.  


In \CAST, send and receive actions were written as:
\ttt{X!x} or \ttt{Y!z}, but in \hac, send and receive arguments must be
enclosed within parenthesis like function call arguments:
\ttt{X!(x)} or \ttt{Y!(z)}.  
Rationale: syntactic consistency and disambiguation with the 
concurrency operator (comma).  



