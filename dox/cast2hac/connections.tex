% "cast2hac/connections.tex"
%	$Id: connections.tex,v 1.2 2006/06/27 02:33:10 fang Exp $

\section{Connections}
\label{sec:connections}

\CAST\ allowed a definition to be connected with fewer port arguments than
port parameters, assuming that the trailing ports remain 
unconnected externally.  
\hac\ forbids this error-prone mismatch, and requires one to pass in
blanks in place of all unconnected ports, even trailing ports.  

For example, given a \defproc (\ttt{define}) \ttt{inv} with three
\bool\ (\ttt{node}) ports and external nodes \ttt{x, y, z}, 
Table~\ref{tab:ports} shows what declaration connections are legal.  

\begin{table}[ht]
\begin{center}
\caption{Examples of legal and illegal port connections}
\label{tab:ports}
\begin{tabular}{|l||c|c|}
\hline
declaration & \CAST & \hac \\ \hline \hline
\ttt{inv x;} & valid & valid \\ \hline
\ttt{inv x();} & valid & invalid \\ \hline
\ttt{inv x(x, y, z);} & valid & valid \\ \hline
\ttt{inv x(x, y);} & valid & invalid \\ \hline
\ttt{inv x(x);} & valid & invalid \\ \hline
\ttt{inv x(x, , z);} & valid & valid \\ \hline
\ttt{inv x(x, , );} & valid & valid \\ \hline
\ttt{inv x(x, y, );} & valid & valid \\ \hline
\end{tabular}
\end{center}
\end{table}

The three invalid \hac\ examples are invalid because
the definition of \ttt{inv} requires exactly three port arguments, 
where only fewer were given.  

