% "cast2hac/loops.tex"
%	$Id: loops.tex,v 1.3 2006/06/27 02:33:10 fang Exp $

\section{Loops}
\label{sec:loops}

In \CAST, the an example of a loop statement in the meta-language
might look like:

\begin{verbatim}
<i:N:
  inv z[i];
  x[i] = y[i];
>
\end{verbatim}

The equivalent in \hac\ would look like:

\begin{verbatim}
(;i:N:
  inv z[i..i];
  x[i] = y[i];
)
\end{verbatim}

The main difference is the use of parentheses instead of 
angle brackets and an extra semicolon operator.  
Notice that the declaration of \ttt{z} in the loop uses
an explicit range to declare each sparse instance 
in the collection (Section~\ref{sec:arrays}).  
(It is highly recommended to keep declarations outside of 
loops where possible, leaving only connection statements inside loops.)

Like in \CAST, loops may be nested arbitrarily deep.  

Do NOT write PRS bodies inside loops, the semantics are currently
undefined.  The PRS sub-language includes a similar loop syntax, 
described in Section~\ref{sec:prs:looprules}.  

