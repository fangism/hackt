% "hackt.tex"
%	$Id: cast2hac.tex,v 1.2 2006/05/26 23:51:26 fang Exp $

\documentclass[12pt]{article}

% for dependency-tracking macros
\input maketexdepend		% ../mk/maketexdepend
\maketexdependignorepattern{maketexdepend}

\usepackage{ifpdf}
\usepackage{verbatim}
\usepackage{makeidx}

% COMPILE OPTIONS
\ifpdf
  \relax
\else
  \let\pdfoutput=\PDFoutput
\fi

\ifpdf
\usepackage[
	draft=false,
	pdftitle={CAST to HAC Migration Guide}, 
	pdfauthor={David Fang}, 
	hyperfigures, 
	hyperindex, 
	bookmarks, 
	colorlinks
	]{hyperref}
\fi

\bibliographystyle{plain}

\makeindex

% PAGE FORMATTING
\setlength{\oddsidemargin}{.5in}
\setlength{\evensidemargin}{0in}
\setlength{\topmargin}{0in}
% \setlength{\topskip}{0in}
% \setlength{\footskip}{0in}
\setlength{\textwidth}{6in}
\setlength{\textheight}{8.5in}

%%%%%%%%%%%%%%%%%%%%%%%%%%%%%%%%%%%%%%%%%%%%%%%%%%%%%%%%%%%%%%%%%%%%%%%%%%%%%%%
% \maketexdependignore{hacdef.tex}
\maketexdependignorepattern{hacdef}
\input hacdef		% from $(srcdir)/../common/hacdef.tex

%%%%%%%%%%%%%%%%%%%%%%%%%%%%%%%%%%%%%%%%%%%%%%%%%%%%%%%%%%%%%%%%%%%%%%%%%%%%%%%
% end of preamble
%%%%%%%%%%%%%%%%%%%%%%%%%%%%%%%%%%%%%%%%%%%%%%%%%%%%%%%%%%%%%%%%%%%%%%%%%%%%%%%
\begin{document}

\title{CAST to HACKT Migration Guide}
\author{David Fang}

% \ifpdf\pdfbookmark[0]{Title Page}{title}\fi
\maketitle

% \ifpdf\pdfbookmark[0]{Table of Contents}{toc}\fi
% \tableofcontents

%%%%%%%%%%%%%%%%%%%%%%%%%%%%%%%%%%%%%%%%%%%%%%%%%%%%%%%%%%%%%%%%%%%%%%%%%%%%%%%
% main contents

\section{Introduction}
\label{sec:intro}

The purpose of this document is to assist \CAST\ users in migrating
old sources to the \hac\ language for use with the \hackt\ tools.  
This document assumes the reader is already familiar with the \CAST\ language
and some of its tools.  
This guide does not cover the new features of the \hac\ language; 
those are covered in the language reference.  

% % % % % % % % % % % % % % % % % % % % % % % % % % % % % % % % % % % % % % % %
\section{Types}
\label{sec:types}

The names of the primitive types have changed.  
Types that correspond to compile-time meta parameters are prefixed with `p', 
such as \pint\ and \pbool.  

\begin{table}
\begin{center}
\caption{Summary of changes in fundmental types}
\label{tab:types}
\begin{tabular}{|c|c|}
\hline
\CAST & \hac \\ \hline \hline
\ttt{node} & \bool \\ \hline
N/A	& \int \\ \hline
\int & \pint \\ \hline
\bool & \pbool \\ \hline
N/A	& \preal \\ \hline
\end{tabular}
\end{center}
\end{table}

The new types \preal\ and \int\ are explained in the \hac\ language reference.  

% % % % % % % % % % % % % % % % % % % % % % % % % % % % % % % % % % % % % % % %
\section{Definitions}
\label{sec:definitions}

Definitions use the keywords \defproc, \defchan, and \defdata.  
The old \CAST\ keyword \ttt{define} can simply be replaced with \defproc.  

% % % % % % % % % % % % % % % % % % % % % % % % % % % % % % % % % % % % % % % %
\section{Arrays}
\label{sec:arrays}

In \CAST, the dimensions were declared between the type-identifier
and the instance-identifier.  
Arrays in \hac\ are syntactically C-style, where the dimensions of the 
array follow the array's identifier.  

% % % % % % % % % % % % % % % % % % % % % % % % % % % % % % % % % % % % % % % %
\section{Templates}
\label{sec:templates}

%%%%%%%%%%%%%%%%%%%%%%%%%%%%%%%%%%%%%%%%%%%%%%%%%%%%%%%%%%%%%%%%%%%%%%%%%%%%%%%
% \ifpdf\pdfbookmark[0]{Bibliography}{bibliography}\fi
% \maketexdependadd{hackt.bib}
% \bibliography{hac}

% \clearpage
% \ifpdf\pdfbookmark[0]{Index}{index}\fi
% \printindex

\end{document}


