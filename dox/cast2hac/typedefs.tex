% "cast2hac/typedefs.tex"
%	$Id: typedefs.tex,v 1.1 2006/06/03 05:52:29 fang Exp $

\section{Typedefs}
\label{sec:typedefs}

Type aliases or \emph{typedefs} were not supported in \CAST, 
but are worth mentioning as a new feature of \hac.  
Like in C, typedefs are a mechanism for giving user-defined names
to an existing type.  
(TODO: discuss the benefits of style.)
If one really wanted to use \ttt{node} and \bool\ as the same type, 
one could write: \ttt{typedef bool node;}
and use \ttt{node} interchangeably with \bool.  

The real benefit is being able to bind definitions templates
to new definitions that just forward template arguments to
underlying types.  

In the library \ttt{channel.hac}, we see the following example:

\begin{verbatim}
template <pint N>
defproc e1of (bool d[N], e) { ... }

typedef	e1of<2> e1of2;
\end{verbatim}

This declarations defines type \ttt{e1of2} to be an alias
to the complete type \ttt{e1of<2>}.  
In \CAST, \ttt{e1of(2)} and \ttt{e1of2} are different definitions and hence,
could not be equivalent types.  
Connecting them required connecting their public port members, 
which was an inconvenience when mixing template types with non-template types.  
More common examples can be found in the library \ttt{env.hac}.  

\subsection{Typedef Templates}
\label{sec:typedefs:templates}

Typedefs themselves may be templated, as best illustrated by 
the following example:

\begin{verbatim}
template <pint N, M>
defproc matrix(bool x[N][M]) { ... }

template <pint L>
typedef matrix<1, L> row;

row<3> a_row_of_length_3;

template <pint H>
typedef matrix<H, 1> col;

col<3> a_col_of_height_3;

template <pint N>
typedef matrix<N, N> square;

square<2> a_2x2_square_matrix;
\end{verbatim}

