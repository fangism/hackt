% "chapters/linkage.tex"

\chapter{Linkage}
\label{sec:linkage}

One of the strengths of the \artxx\ language is modularity.  
The old implementation of the CAST \index{CAST} language 
was a single-pass interpreter.  
The current implementation of an \artxx\ compiler allows
one to compile modules independently and later link modules together
into a coherent object file.  
Modular compilation leads to efficient recompilation, 
library development...

%%%%%%%%%%%%%%%%%%%%%%%%%%%%%%%%%%%%%%%%%%%%%%%%%%%%%%%%%%%%%%%%%%%%%%%%%%%%%%%
\section{Visibility}
\label{sec:linkage:visibility}

Definitions and instantiations within a compilation module can either be
publicly accessible to other modules, or private and inaccessible.  
By default, all entities are public, i.e. their uses are \emph{exported}.  
To make an entity private, simply prefix the first declaration
or prototype with the keyword \ttt{static}, like in C.  
To refer to an entity defined in another module, 
simply prefix a declaration with the keyword \ttt{extern}, like in C.  

Implementation:
Generate automatic headers from implementation files.  

%%%%%%%%%%%%%%%%%%%%%%%%%%%%%%%%%%%%%%%%%%%%%%%%%%%%%%%%%%%%%%%%%%%%%%%%%%%%%%%
\section{Ordering}
\label{sec:linkage:ordering}



%%%%%%%%%%%%%%%%%%%%%%%%%%%%%%%%%%%%%%%%%%%%%%%%%%%%%%%%%%%%%%%%%%%%%%%%%%%%%%%
\section{Questions}
\label{sec:linkage:questions}

How does linkage apply to typedefs?
Can one make a definition static, but a typedef thereof exported?
(Could be useful for simplifying interfaces to definitions...)

%%%%%%%%%%%%%%%%%%%%%%%%%%%%%%%%%%%%%%%%%%%%%%%%%%%%%%%%%%%%%%%%%%%%%%%%%%%%%%%

