% "chapters/types.tex"

\chapter{Types}
\label{sec:types}

In \artxx, there are four classes of types to use:
parameters, datatypes, channels, and processes.  

%%%%%%%%%%%%%%%%%%%%%%%%%%%%%%%%%%%%%%%%%%%%%%%%%%%%%%%%%%%%%%%%%%%%%%%%%%%%%%%
\section{Parameters}
\label{sec:parameters}

Parameters differ from the others in that they don't correspond
to any physical instances; they are merely user-manipulated values.  
Currently, there are only two (built-in) parameter types: 
\pint, which is an integer value,
and \pbool, which is a boolean value.  

Parameters may be set no more than once, and they must be 
initialized before they are used.  
Using uninitialized parameters is an error.  


We cover templates more in-depth in Chapter blah.  
ART++ supports parametric types, or templates, much like C++. 
Terminology: a template definition (or just template)
is a parameterized definition.  
User-defined datatypes, channels, and processes may be parameterized.  
Currently the only types that may appear in template signatures
are parameters and arrays thereof.  
Later we may add support for template template arguments, muahahaha.  

%%%%%%%%%%%%%%%%%%%%%%%%%%%%%%%%%%%%%%%%%%%%%%%%%%%%%%%%%%%%%%%%%%%%%%%%%%%%%%%
\section{Definitions}
\label{sec:types:definition}

A \emph{type} can either be a reference to a non-template definition
(which may be built-in) or a reference to a template definition
with a set of parameters.  

Two (non-template) types are equivalent if and only if 
they refer to the same non-template definition.  
In Chapter~\ref{sec:templates} we extend the notion of type-equivalence
to cover types with templates.  

%%%%%%%%%%%%%%%%%%%%%%%%%%%%%%%%%%%%%%%%%%%%%%%%%%%%%%%%%%%%%%%%%%%%%%%%%%%%%%%
\section{Completeness and Usability}
\label{sec:types:completeness}

In \artxx, there are three levels of completeness for definitions.  

\begin{enumerate}
\item A definition is \emph{declared}
	if its template signature (which may be empty) has been defined.  
	Forward declarations of a type only specify the name
	and template parameters, and no other information.  
\item A definition is \emph{signed} if it is declared 
	and its public ports interface (which may be empty) has been defined.  
\item A definition is \emph{defined} if it is signed
	and its body has been defined.  
\end{enumerate}

Since type information only refers to template parameters, 
a type may refer to any declared definition.  

% % % % % % % % % % % % % % % % % % % % % % % % % % % % % % % % % % % % % % % %
\subsection{Instantiation}
\label{sec:types:completeness:instantiation}

An instantiation is the creation of 

\subsection{Connection}
\label{sec:types:completeness:connection}

See Chapter~\ref{sec:connections}.  

\subsection{Typedefs}
\label{sec:types:completeness:typedefs}

See Chapter~\ref{sec:typedefs}.  

%%%%%%%%%%%%%%%%%%%%%%%%%%%%%%%%%%%%%%%%%%%%%%%%%%%%%%%%%%%%%%%%%%%%%%%%%%%%%%%
\section{Future extensions}
\label{sec:types:future}

Support \ttt{typeof()} operator and \ttt{sizeof()} operator.  


