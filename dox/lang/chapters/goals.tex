% "chapters/goals.tex"

\chapter{Goals}
\label{sec:goals}

Designing a new language is not a task for the faint-hearted.  
There has to be sufficient motivation to justify such labor:

\begin{itemize}
\item No existing language (or composition thereof)
	meets the requirements demanded.  
\item A workaround atop of existing languages is deemed to be insufficient.
\item For fun.  
\end{itemize}

As selfish as it may seem, the \artxx\ language is designed to meet a very
specific requirement of the author.  
The author's desire is to provide a language to facilitate:

\begin{itemize}
\item Ease and effectiveness of (asynchronous) VLSI design automation
\item Ease of automatic design space exploration 
	(at high architecture level and low level circuits)
\end{itemize}

%%%%%%%%%%%%%%%%%%%%%%%%%%%%%%%%%%%%%%%%%%%%%%%%%%%%%%%%%%%%%%%%%%%%%%%%%%%%%%%
\section{Design Automation}
\label{sec:goals:EDA}

What do other languages lack?
Asynchrony?
Heirarchical information.  

When charged with the task of designing a circuit component, 
one is usually given a functional specification to meet.  
The difficulty often lies with coming up with fitting 
functional specification ---
knowing \emph{a priori} the context in which a component is used.  
Without the context in which a component is used, 
it is futile to optimize the design of that component.  
Often, one functional specification for a component is really 
meant to be used in multiple contexts, in which case, 
one would design a different version for each context.  

``Heavy-tree'' of definition uses (like calling context stacks).  

Ramble ramble ramble...

Shape and form factors...

Optimization using dynamic performance information...

%%%%%%%%%%%%%%%%%%%%%%%%%%%%%%%%%%%%%%%%%%%%%%%%%%%%%%%%%%%%%%%%%%%%%%%%%%%%%%%
\section{Design Space Exploration}
\label{sec:goals:explore}

The ultimate ambition for this project is to be able to 
automatically explore the design space of implementations of
significantly complex functional specifications.  

Program transformations.  

Design choices fall into two major categories:
1) Quantitative (How many?  How wide?  How large?) and
2) Qualitative (What \emph{kind}?).  

Quantitative parameters are simply traits that can be described
quantitatively:  How many buffers?  How wide a datapath?  

Examples of qualitative parameters:
What kind of buffer?  What protocol or reshuffling?  
Which kind of adder?  Whether or not to use a speculative path?
Whether or not to introduce resource arbitration?
Parallel or serial?

Objectives.
Area, energy, performance, efficiency.  

%%%%%%%%%%%%%%%%%%%%%%%%%%%%%%%%%%%%%%%%%%%%%%%%%%%%%%%%%%%%%%%%%%%%%%%%%%%%%%%
\section{Asynchrony}
\label{sec:goals:async}

So what does \emph{asynchronous} VLSI have to do with deciding
to use a new language?  

Verilog and VHDL shortcomings...
Other existing (public) async. synthesis tools.  

%%%%%%%%%%%%%%%%%%%%%%%%%%%%%%%%%%%%%%%%%%%%%%%%%%%%%%%%%%%%%%%%%%%%%%%%%%%%%%%

