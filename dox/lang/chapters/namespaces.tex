% "chapters/namespaces.tex"

%	$Id: namespaces.tex,v 1.3.116.1 2005/12/13 03:58:24 fang Exp $

\chapter{Namespaces}
\label{sec:namespaces}

To avoid name collisions, \hac\ supports namespaces.  
Namespaces are also useful for organizing closely related definitions
and instantiations.  
\hac\ allows namespaces to be nested, although one-level (flat)
namespaces may be emulated by name-mangling.  

The default namespace is the \emph{global} namespace, 
the ultimate root of all user-defined namespaces.  
The global namespace has no parent namespace.  

See Appendix C.10 (Technicalities:Namespaces) of 
Stroustrup's \textit{The C++ Programming Language}.  

%%%%%%%%%%%%%%%%%%%%%%%%%%%%%%%%%%%%%%%%%%%%%%%%%%%%%%%%%%%%%%%%%%%%%%%%%%%%%%%
\section{Identifiers}
\label{sec:namespaces:identifiers}

Identifiers may be prefixed with namespaces using the 
scope (\ttt{::}) operator, e.g. \ttt{zoo::aquatic::fish}.  
Any identifier that begins with \ttt{::} is an absolute identifier, 
one whose namespace path is specified from the global namespace.  
Identifiers fall into one of several categories:

\begin{itemize}
\item unqualified, in which the parent namespace is implicit
\item qualified-relative, where an identifier is prefixed with a
	partial namespace path (e.g. \ttt{dinner::food::salad})
\item qualified-absolute, where the full path to a given entity is 
	specified in the identifier's prefix.  
	(e.g. \ttt{::plane} or \ttt{::farm::horse})
\end{itemize}


%%%%%%%%%%%%%%%%%%%%%%%%%%%%%%%%%%%%%%%%%%%%%%%%%%%%%%%%%%%%%%%%%%%%%%%%%%%%%%%
\section{Importing}
\label{sec:namespaces:import}

To make all public definitions within a namespace available
using unqualified identifiers, one can import an entire namespace.  

Aliasing is like psuedo subnamespace.

Each time a namepace is closed, all of its imported namespaces are
discarded, which means that the next time that namespace is opened, 
it loses its previous imports.  
This can be useful for resetting imports, should the need or desire arise.  

Since the global namespace cannot be closed, 
import directives in the global namespace last until the end-of-file.  


%%%%%%%%%%%%%%%%%%%%%%%%%%%%%%%%%%%%%%%%%%%%%%%%%%%%%%%%%%%%%%%%%%%%%%%%%%%%%%%
\section{Resolution}
\label{sec:namespaces:resolution}

In this section, we describe the order in which namespaces are
searched to resolve identifiers.  
Given an unqualified identifier, references are resolved as follows:

\begin{enumerate}
\item Lookup the identifier in the current namespace 
	(where the reference is made).  
	If a match is found, it is guaranteed to be unique, 
	otherwise, continue searching.  
\item Lookup the identifier in each imported and aliased namespace.  
	If a unique match is found, return it.  
	In the case of multiple matches (in different namespaces), 
	report ambiguity as an error.  
	If no matches are found, continue searching.
\item If not already in the global namespace, 
	continue searching using steps 1 and 2 in the \emph{parent} namespace.  
	Return the first unique match found or an error if zero
	or more than one match is found.  
\end{enumerate}

%%%%%%%%%%%%%%%%%%%%%%%%%%%%%%%%%%%%%%%%%%%%%%%%%%%%%%%%%%%%%%%%%%%%%%%%%%%%%%%
\section{Issues}
\label{sec:namespaces:issues}

Interpretation:

What does it mean for a parameter assignment or connection to 
appear in a namespace?  
Doesn't make any sense.  
Therefore forbid assignments and connections (actions) inside namespaces.  
Namespaces may include instantiations and definitions and other namespaces.  

Implementation:

