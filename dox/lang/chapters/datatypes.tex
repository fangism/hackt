% "chapters/datatypes.tex"

%	$Id: datatypes.tex,v 1.6 2006/11/15 21:56:44 fang Exp $

\chapter{Datatypes}
\label{sec:datatypes}

Datatypes are physical representations of information.  
Or not.  

TODO: this must spell out structs and user-defined datatype implementations
as distinct notions.  


%%%%%%%%%%%%%%%%%%%%%%%%%%%%%%%%%%%%%%%%%%%%%%%%%%%%%%%%%%%%%%%%%%%%%%%%%%%%%%%
\section{Built-in datatypes}
\label{sec:datatypes:builtin}

Currently, \hac\ has two built in datatypes: \bool\ and \int.  
These are not to be confused with the parameter types \pbool\ and \pint.  
A \bool\ represents the state of a physical or logical node.  
An \int\ is simply an array of \bool{}s with an integer interpretation.  

% % % % % % % % % % % % % % % % % % % % % % % % % % % % % % % % % % % % % % % %
\subsection{Booleans}
\label{sec:datatypes:builtin:bool}

In the nonmeta language, assigning a \pbool\ value to a \bool\ 
is common and legal.  
The compiler should eventually resolve all meta-parameter values to
constants, which would result in assigning a boolean constant
to a \bool\ data instance.  

In the following example, the \ttt{constant<true>} and
\ttt{constant<false>} types would unroll as one-time
assignments to different values.  
(See Chapter~\ref{sec:chp} for the description of the CHP language.)

\begin{verbatim}
template <pbool B>
defproc constant(bool b) {
  chp { b := B }
}
\end{verbatim}

Nonmeta operators: 
e.g. in CHP.  

\tbf{Defect}: Currently \bool\ is overloaded to mean different things
in different contexts.  It is (tentatively) both a physical type
representing a single node or net, and abstract data type representing
a boolean value (whose implementation may be multi-node, such as dual-rail.)
In a data type context, \bool\ is synonymous with \ttt{int<1>}.  
There is an ambiguity with \ttt{defproc foo(bool b) \{...\}}, 
where it is not known without additional context whether \bool\ refers to 
the physical type or the data type.  
Until implementation-interface semantics are better defined and implemented, 
we must allow CHP to manipulate physical \bool{}s directly.  
Eventually, we would like an unambiguous use of \bool, which may require
introducing a typedef for \ttt{int<1>}, such as \ttt{dbool}, 
either built-in or library-defined.  

% % % % % % % % % % % % % % % % % % % % % % % % % % % % % % % % % % % % % % % %
\subsection{Integers}
\label{sec:datatypes:builtin:int}

The \int\ type can take an optional \emph{width} parameter that specifies
the \emph{physical} number of bits used to represent the integer.  
An integer value need not necessarily be encoded in two's-complement; 
one may use more bits to encode more abstract values (like ``not-an-number''), 
or employ error-correcting codes.  
The default width of an \int\ is 32\footnote{32 was chosen arbitrarily.}.
Technically speaking, \int\ is a built-in templated (parameterized) 
datatype definition.  
Templates are discussed in more detail in Chapter~\ref{sec:templates}.  
The specify an \int's width, one can write \ttt{int<}\tit{pint}\ttt{>}.  

For integer type-checking in the nonmeta language, 
the assignment of a \pint\ value or constant to an \int\ of
\emph{any} width is legal.  
Implementation detail: this is accomplished by using \ttt{int<0>}
as a magic width type for meta-valued integers and constants.  

Arithmetic and relational operators:
In CHP, the standard arithmetic operations on \int\ types interprets
the bits as signed two's-complement integers.  
Operator overloading is not yet supported for user-defined datatypes, 
but may be in the future.  

Update: \ttt{int<1>} is equivalent to \ttt{bool} in a data type context.  


%%%%%%%%%%%%%%%%%%%%%%%%%%%%%%%%%%%%%%%%%%%%%%%%%%%%%%%%%%%%%%%%%%%%%%%%%%%%%%%
\section{Enumerations}
\label{sec:datatype:enum}

Enumerations are sets of values associated with user-specified names.  
The value members of an enumeration represent a set of 
logical values that only have meaning in the enumeration's context, 
i.e. they are not publicly observable values.  
(This is unlike C, in which enumerations can take integer values that 
can be passed to and from integer variables.)
Enumerations are particularly useful for specifying 
control and data interfaces between communicating processes.  
Think of enumerations as tags that can be understood by 
the sender and receiver of the enumerated type.  

The only values that an enumerated instance can take are 
those specified in the enumerated type.  
Thus one can never assign an integer or boolean value to an enumeration, 
nor can one assign an enumerated value to an \int\ or \bool\ or
other user-defined type.  
One can only compare enumerated values of the same enumerated type.  

There is no notion of equivalence between enumerated types 
(outside of typedefs, Ch.~\ref{sec:typedefs}).  

%%%%%%%%%%%%%%%%%%%%%%%%%%%%%%%%%%%%%%%%%%%%%%%%%%%%%%%%%%%%%%%%%%%%%%%%%%%%%%%
\section{User-defined datatypes}
\label{sec:datatype:userdef}

In \hac, one can define arbitrarily complex datatypes.  
User-defined datatypes resemble structs in C.  

Actually, the user-defined datatype implementations are
NOT the same as plain structs.  
They describe an \emph{implementation} of an abstract data type
with physical types.  

% % % % % % % % % % % % % % % % % % % % % % % % % % % % % % % % % % % % % % % %
\subsection{Declarations}
\label{sec:datatype:userdef:declaration}

% % % % % % % % % % % % % % % % % % % % % % % % % % % % % % % % % % % % % % % %
\subsection{Definitions}
\label{sec:datatype:userdef:definition}

% % % % % % % % % % % % % % % % % % % % % % % % % % % % % % % % % % % % % % % %
\subsection{Views}
\label{sec:datatype:userdef:views}

Views are a way of sub-typing datatypes.  

Views are simply specific interpretations or refinements of a datatype.  



%%%%%%%%%%%%%%%%%%%%%%%%%%%%%%%%%%%%%%%%%%%%%%%%%%%%%%%%%%%%%%%%%%%%%%%%%%%%%%%

