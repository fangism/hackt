% "chapters/datatypes.tex"

\chapter{Datatypes}
\label{sec:datatypes}

Datatypes are physical representations of information.  
Or not.  


%%%%%%%%%%%%%%%%%%%%%%%%%%%%%%%%%%%%%%%%%%%%%%%%%%%%%%%%%%%%%%%%%%%%%%%%%%%%%%%
\section{Built-in datatypes}
\label{sec:datatypes:builtin}

Currently, \artxx\ has two built in datatypes: \bool\ and \int.  
These are not to be confused with the parameter types \pbool\ and \pint.  
A \bool\ represents the state of a physical or logical node.  
An \int\ is simply an array of \bool s with an integer interpretation.  

The \int\ type can take an optional \emph{width} parameter that specifies
the \emph{physical} number of bits used to represent the integer.  
An integer value need not necessarily be encoded in two's-complement; 
one may use more bits to encode more abstract values (like ``not-an-number''), 
or employ error-correcting codes.  
The default width of an \int\ is 32\footnote{32 was chosen arbitrarily.}.
Technically speaking, \int\ is a built-in templated (parameterized) 
datatype definition.  
Templates are discussed in more detail in Chapter~\ref{sec:templates}.  
The specify an \int's width, one can write \ttt{int<}\tit{pint}\ttt{>}.  

In CHP, the standard arithmetic operations on \int\ types interprets
the bits as signed two's-complement integers.  
Operator overloading is not yet supported for user-defined datatypes, 
but may be in the future.  

%%%%%%%%%%%%%%%%%%%%%%%%%%%%%%%%%%%%%%%%%%%%%%%%%%%%%%%%%%%%%%%%%%%%%%%%%%%%%%%
\section{Enumerations}
\label{sec:datatype:enum}

Enumerations are sets of values associated with user-specified names.  
The value members of an enumeration represent a set of 
logical values that only have meaning in the enumeration's context, 
i.e. they are not publicly observable values.  
(This is unlike C, in which enumerations can take integer values that 
can be passed to and from integer variables.)
Enumerations are particularly useful for specifying 
control and data interfaces between communicating processes.  
Think of enumerations as tags that can be understood by 
the sender and receiver of the enumerated type.  

The only values that an enumerated instance can take are 
those specified in the enumerated type.  
Thus one can never assign an integer or boolean value to an enumeration, 
nor can one assign an enumerated value to an \int\ or \bool\ or
other user-defined type.  
One can only compare enumerated values of the same enumerated type.  

There is no notion of equivalence between enumerated types 
(outside of typedefs, Ch.~\ref{sec:typedefs}).  

%%%%%%%%%%%%%%%%%%%%%%%%%%%%%%%%%%%%%%%%%%%%%%%%%%%%%%%%%%%%%%%%%%%%%%%%%%%%%%%
\section{User-defined datatypes}
\label{sec:datatype:userdef}

In \artxx, one can define arbitrarily complex datatypes.  
User-defined datatypes resemble structs in C.  

% % % % % % % % % % % % % % % % % % % % % % % % % % % % % % % % % % % % % % % %
\subsection{Declarations}
\label{sec:datatype:userdef:declaration}

% % % % % % % % % % % % % % % % % % % % % % % % % % % % % % % % % % % % % % % %
\subsection{Definitions}
\label{sec:datatype:userdef:definition}

% % % % % % % % % % % % % % % % % % % % % % % % % % % % % % % % % % % % % % % %
\subsection{Views}
\label{sec:datatype:userdef:views}

Views are a way of sub-typing datatypes.  

Views are simply specific interpretations or refinements of a datatype.  



%%%%%%%%%%%%%%%%%%%%%%%%%%%%%%%%%%%%%%%%%%%%%%%%%%%%%%%%%%%%%%%%%%%%%%%%%%%%%%%

