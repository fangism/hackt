% "chapters/connections.tex"

%	$Id: connections.tex,v 1.6 2005/06/22 22:13:27 fang Exp $

\chapter{Connections}
\label{sec:connections}

Connections are a relation between \emph{instance references}.  
(Connections are established in the meta-language processing 
of compilation; they are determined at compile-time only.)

Instance references refer to specific entities at unroll-time.  
Instance references fall into two categories: implicit and explicit.  
See Chapter~\ref{sec:arrays}.  

What does \verb|a = b| mean in the namespace or definition context?

\begin{itemize}
\item Parameters: the value of b is assigned to a.
	This is only valid if \verb|b| is instantiated and initialized, 
	and \verb|a| is instantiated and uninitialized.  
\item Datatypes (both built-in and user-defined), channels and processes:
	\verb|a| and \verb|b| refer to the same instance, in other words, 
	they are \emph{aliases}.
\end{itemize}

If the types are user-defined, then aliasing is recursive.  
For example, if the type of \verb|a| and \verb|b| has 
members (either public or private) \verb|x| and \verb|y| internally aliased, 
the \verb|a.x|, \verb|a.y|, \verb|b.x|, and \verb|b,y| are all 
valid references to the same instance of \verb|x| and \verb|y|'s type.  
(Implementation:
This can simply be accomplished by mapping \verb|a| and \verb|b|
to the same instance, saving the trouble of recursive aliasing, 
and generating the combinations of names, 
not that that is ever a problem.)

Since connections and aliases are unrolled, the actual instance objects 
are not created until all connections have been processed.  

Compiler options (proposed to support):
\begin{itemize}
\item \verb|-Wprocess-alias|:
	warning for connections between process
	(since the semantics seem arbitrary at this point and 
	are prone to future change), 
\item \verb|-Wchannel-connections|: 
	warning for suspicious wrong connections with channels
	(multiple senders or multiple receivers)
\end{itemize}


Support for non-alias connections?

