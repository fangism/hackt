% "hac.tex"

%	$Id: hac.tex,v 1.7 2006/11/26 04:41:06 fang Exp $

\documentclass[12pt]{book}

% for dependency-tracking macros
% \input maketexdepend		% ../mk/maketexdepend
% \maketexdependignorepattern{maketexdepend}

\usepackage{ifpdf}
\usepackage{verbatim}
\usepackage{makeidx}
\usepackage{amsthm}		% AMS-theorem

% COMPILE OPTIONS
\ifpdf
  \relax
\else
  \let\pdfoutput=\PDFoutput
\fi

\ifpdf
\usepackage[
	draft=false,
	pdftitle={The HAC Language}, 
	pdfauthor={David Fang}, 
	hyperfigures, 
	hyperindex, 
	bookmarks, 
	colorlinks
	]{hyperref}
\fi

\bibliographystyle{plain}

\makeindex

% PAGE FORMATTING
\setlength{\oddsidemargin}{.5in}
\setlength{\evensidemargin}{0in}
\setlength{\topmargin}{0in}
% \setlength{\topskip}{0in}
% \setlength{\footskip}{0in}
\setlength{\textwidth}{6in}
\setlength{\textheight}{8.5in}

% more sophisticated declaration of example environment
\newtheoremstyle{sectionexample}{\topsep}{\topsep}%
{\ttfamily}%	Body font
{}%	Indent amount (empty = no indent, \parindent = para indent)
{\bfseries}%	Thm head font
{}%	Punctuation after thm head
{\newline}%	Space after thm head (\newline = linebreak)
{\thmname{#1}\thmnumber{ #2}\thmnote{ #3}}%	Thm head spec
\theoremstyle{sectionexample}
\newtheorem{sectionexample}{Example}[section]

% kind of dirty space-hacking, but modular
\newenvironment{example}[1]
  {\begin{sectionexample}[#1]\ \\ \vspace{-2.5em}\begin{quote}}
  {\end{quote}\end{sectionexample}}

%%%%%%%%%%%%%%%%%%%%%%%%%%%%%%%%%%%%%%%%%%%%%%%%%%%%%%%%%%%%%%%%%%%%%%%%%%%%%%%
% \maketexdependignore{hacdef.tex}
% \maketexdependignorepattern{hacdef}
\input hacdef		% from $(srcdir)/../common/hacdef.tex

%%%%%%%%%%%%%%%%%%%%%%%%%%%%%%%%%%%%%%%%%%%%%%%%%%%%%%%%%%%%%%%%%%%%%%%%%%%%%%%
% end of preamble
%%%%%%%%%%%%%%%%%%%%%%%%%%%%%%%%%%%%%%%%%%%%%%%%%%%%%%%%%%%%%%%%%%%%%%%%%%%%%%%
\begin{document}

\title{The HAC Language}
\author{David Fang}

\ifpdf\pdfbookmark[0]{Title Page}{title}\fi
\maketitle

% trying to correct odd even margins...
\begin{comment}
\clearpage
\centerline{This page is intentionally left blank.}
\clearpage
\centerline{So is this page}
\clearpage
\end{comment}

\ifpdf\pdfbookmark[0]{Table of Contents}{toc}\fi
\tableofcontents

%%%%%%%%%%%%%%%%%%%%%%%%%%%%%%%%%%%%%%%%%%%%%%%%%%%%%%%%%%%%%%%%%%%%%%%%%%%%%%%

\input chapters/preface
\input chapters/goals
\input chapters/intro
\input chapters/types
\input chapters/expressions
\input chapters/arrays
\input chapters/processes
\input chapters/channels
\input chapters/datatypes
\input chapters/namespaces
\input chapters/templates
% \input chapters/instances
\input chapters/connections
\input chapters/typedefs
% \input chapters/control	% loops and conditionals
\input chapters/linkage
\input chapters/CHP
\input chapters/PRS
\input chapters/SPEC

%%%%%%%%%%%%%%%%%%%%%%%%%%%%%%%%%%%%%%%%%%%%%%%%%%%%%%%%%%%%%%%%%%%%%%%%%%%%%%%

\ifpdf\pdfbookmark[0]{Bibliography}{bibliography}\fi
% \maketexdependadd{hac.bib}
\bibliography{hac}

\clearpage
\ifpdf\pdfbookmark[0]{Index}{index}\fi
\printindex

\end{document}


